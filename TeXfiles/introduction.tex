\chapter{Introduction}
\label{chap:introduction}



\section{Topic}
Task and time management applications help users stay organized by keeping track of notes, meetings, tasks etc. Many applications exist that are directed towards this purpose, ranging from simple note-taking applications and to-do lists to more advanced ones like calendars and scheduling applications. A calendar typically holds events or appointments for a user while a to-do list will keep more detailed and lower level tasks. All of these applications have in common that they relieve the user of having to remember things, thus allowing the user to focus more deeply on other things.

Context awareness is also a field of research that has received more attention in recent years. The reason for this is the increasing number of mobile devices that are available and also the increasing functionality of these devices. When it comes to context awareness, the sensors in mobile devices play a huge role. There are more sensors packed into these devices now than ever before, which allows applications to collect very specific types of contextual information. This in turn allows for the development of applications that have very specific and tailored purposes.

By combining context awareness and task management, we open up for new types of applications. Smarter systems could be built that leverage contextual information, both past and present, to adapt its behavior to accommodate very specific situations. A calendar system for example, could evaluate a user's upcoming meeting and the current location of the user. Taking into account the distance between the user and meeting location, the system could then deliver a reminder at the appropriate time, allowing the user to catch the meeting. Another example would be a to-do list application that could leverage contextual information about previously performed tasks to provide task recommendations to the user.





\section{Problem description}
Task and time management applications are valuable tools that are used by many people. These applications can be especially valuable on mobile devices as the users can carry these devices with them, thus having the application data easily accessible. Concrete and popular examples of such applicatios are calendars such as Google Calendar\cite{googlecalendar} or to-do list applications such as Trello\cite{trello} or Todoist\cite{todoist}. Even though these, and many similar applications are very useful, their functionality and usefulness could be further improved by integrating user context. 

Utilization of contextual information in different scenarios have been widely researched. However, we have found no research that studies the usage of such information in the specific domain of task and time management systems. Many systems are improved by making them context-aware. One example of this is Learning Management Systems (LMS) that can suggest learning content based on the users current context, and in doing so increasing the amount of learning for the user. Following examples such as this, it is believed that task management applications can also reap the benefits of context awareness.

This study will look into the area of utilizing contextual information in task management applications.





\section{Research questions}

This thesis is three folded. Firstly, there is a context acquisition and representation domain. Secondly, we have the software engineering domain, which relates to application design and functionality. The third domain revolves around the context in which the previous two are placed for this thesis, namely students and the educational context for which the thesis is conducted in. The proposed solution will have to address all of these domains in detail. Therefore, these research questions have been formed for each of the areas:
\begin{itemize}
	\item What contexts are relevant in a task management application, and how should they be acquired and modeled?
	\item How can a task management application be designed that allows it to utilize contextual information?  
	\item Research question 3?
\end{itemize}

\section{Thesis structure}