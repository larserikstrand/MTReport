\chapter{Introduction}
\label{chap:introduction}



\section{Topic}
Task and time management applications help users stay organized by keeping track of notes, meetings, tasks etc. Many applications exist that are directed towards this purpose, ranging from simple note-taking applications and to-do lists to more advanced ones like calendars and scheduling applications. A calendar typically holds events or appointments for a user while a to-do list will keep more detailed and lower level tasks. All of these applications have in common that they relieve the user of having to remember things, thus allowing the user to focus more deeply on other things.

Context awareness is also a field of research that has received more attention in recent years. The reason for this is the increasing number of mobile devices that are available and also the increasing functionality of these devices. When it comes to context awareness, the sensors in mobile devices play a huge role. There are more sensors packed into these devices now than ever before, which allows applications to collect very specific types of contextual information. This in turn allows for the development of applications that have very specific and tailored purposes.

By combining context awareness and task management, we open up for new types of applications. Smarter systems could be built that leverage contextual information, both past and present, to adapt its behavior to accommodate very specific situations. A calendar system for example, could evaluate a user's upcoming meeting and the current location of the user. Taking into account the distance between the user and meeting location, the system could then deliver a reminder at the appropriate time, allowing the user to catch the meeting. Another example would be a to-do list application that could leverage contextual information about previously performed tasks to provide task recommendations to the user.





\section{Problem description}
Task and time management applications are valuable tools that are used by many people. These applications can be especially valuable on mobile devices as the users can carry these devices with them, thus having the application data easily accessible. Concrete and popular examples of such applications are calendars such as Google Calendar\cite{googlecalendar} or to-do list applications such as Trello\cite{trello} or Todoist\cite{todoist}. Even though these, and many similar applications are very useful, their functionality and usefulness could be further improved by integrating user context. 

Utilization of contextual information in different scenarios have been widely researched. However, we have found no research that studies the usage of such information in the specific domain of task and time management systems. Many systems are improved by making them context-aware. One example of this is Learning Management Systems (LMS) that can suggest learning content based on the users current context\cite{verbert2012context}, and in doing so increasing the level of learning for the user. Following examples such as this, it is believed that task management applications can also reap the benefits of context awareness.

This study will look into the area of utilizing contextual information in task management applications. We will design a proof of concept application that tries to make use of the user's contexts.

\section{Justification, motivation and benefits}
The complete envisioned application is a personal information manager (PIM) where the user only needs to inform the application of the tasks that he/she needs to perform. The application would then suggest a current task and provide an optimal ordering of the other tasks, taking into account the current user context. This optimization can for example be based on minimizing total time spent completing the tasks, minimizing total travel distance between tasks, thereby reducing travel costs and pollution, or a combination of these. With respect to time, this would mean that completing the tasks in any other order than that suggested by the system would lead to a longer time spenditure. In order to create such a system, many areas would need extensive research, more than that which is possible to complete within the scope of this thesis. However, sub-parts of the system can be identified and researched, providing building blocks for the realization of the entire system. These sub-parts include how to collect relevant context data, how to store those data, how to implement an overall design of such an application and how to evaluate past tasks and contexts to successfully produce task suggestions. Identifying and researching these parts of the system justifies the work to be conducted in this thesis, as it allows for working towards the higher-end goal. The motivation behind the thesis is to be able to contribute to to-do applications, and by extension task management applications in general, by further enhancing their task management capabilities.



\section{Research questions}

This thesis is three folded. Firstly, there is the area of context collection and representation. Secondly, we have the software engineering area, which relates to application design and functionality. The third area revolves around the context in which the previous two are placed for this thesis, namely students and the educational context for which the thesis is conducted in. The proposed solution will have to address all of these areas in detail. Therefore, these research questions have been formed for each of the areas:
\begin{itemize}
	\item \textbf{Research question 1}: What contexts are relevant in a task management application, and how can they be acquired and modeled?
	\item \textbf{Research question 2}: How can a task management application be designed and what are the important design decisions when making such an application?
	\item \textbf{Research question 3}: How can context information be utilized in a task management application?
\end{itemize}

\section{Thesis structure}
Chapter~\ref{chap:relatedwork} provides the necessary background that related to the work in this thesis. State-of-the-art task management applications are also covered here. Chapter~\ref{chap:methodology} describes the whole process of designing the to-do list application and is also in its entirety the answer to research question 2 and 3. Chapter~\ref{chap:results} presents the findings in this thesis and is together with Chapter \ref{chap:methodology} and \ref{chap:discussion} the answer to research question 1. Chapter~\ref{chap:discussion} discusses the different approaches and decisions made in this thesis, as well as discussing alternatives for some of these decisions. The work is then concluded in Chapter~\ref{chap:conclusion} and futer work is discussed in Chapter~\ref{chap:futurework}.